\documentclass{article}
\usepackage[greek,english]{babel}
\usepackage[utf8]{inputenc}
\usepackage{verse}
\usepackage{marginnote}
\title{Note sull'agone dell'Antigone

vv. 441-470}

\author{Zeno De Cesare }
\date{Dicembre 2022}
\usepackage[
backend=biber,
style=alphabetic,
sorting=ynt
]{biblatex}
\addbibresource{sample.bib}

\begin{document}

\maketitle
\section{L'agone nel secondo episodio dell'Antigone}
As Griffith so insightfully points out, this second episodion (376–581) is composed of three increasingly revelatory phases. In the first phase (376–445), the watchman enters leading Antigone, relates how they caught her re-burying the body, and leaves a free man. In the second phase (446–525), we have the exchange between Creon and Antigone, which escalates the tension of the play with the rapidly growing hostility between Antigone and her uncle, and the beginning of the disintegration of Creon’s mental composure. In the third phase (526-81), Ismene is brought in, and a three-way dialogue ensues, which puts an end to the possibility of any resolution\footnote{\cite{Griffith}}.
\selectlanguage{greek}
\par
\begin{verse}
\poemlines{5}
\setverselinenums{441}{445}
\flagverse{Κρέων} σὲ δή, σὲ τὴν νεύουσαν εἰς πέδον κάρα,\footnote{\textbf{νεύουσαν}\selectlanguage{english} Antigone rivolge sdegnosamente lo sguardo a terra. Non ha risposto alle parole del coro e non è intervenuta nella conversazione tra la guardia e Creonte. Verosimilmente l'eroina alzerà lo sguardo nel formulare la sua risposta, con efficace azione scenica.}  \\
φὴς ἢ καταρνεῖ μὴ δεδρακέναι τάδε: \\
\flagverse{Ἀντιγόνη} καὶ φημὶ δρᾶσαι κοὐκ ἀπαρνοῦμαι τὸ μή. \\
\flagverse{Κρέων} σὺ μὲν κομίζοις ἂν σεαυτὸν ᾖ θέλεις \\
ἔξω βαρείας αἰτίας ἐλεύθερον: \\
σὺ δ᾽ εἰπέ μοι μὴ μῆκος, ἀλλὰ συντόμως, \\
ᾔδησθα κηρυχθέντα μὴ πράσσειν τάδε; \\
\flagverse{Ἀντιγόνη}  ᾔδη: τί δ᾽ οὐκ ἔμελλον; ἐμφανῆ γὰρ ἦν. \\
\flagverse{Κρέων} καὶ δῆτ᾽ ἐτόλμας τούσδ᾽ ὑπερβαίνειν νόμους; \\
\flagverse{Ἀντιγόνη}\footnote{ \selectlanguage{english}“This speech is one of the most famous in all Greek tragedy. With its forthright espousal of unwritten = divine = natural ‘laws,’ as against written (or proclaimed) = human = civil legislation, it has been quoted in support of countless acts of disobedience and rebellion against governments of all kinds” (p. 199). Jebb’s examination of Antigone’s claim is interesting. He begins his comment (450–53) by saying that “Zeus is opposed to Creon’s edicts, not only as
supreme god and therefore guardian of all religious duty, but also in each of his two special qualities, as \selectlanguage{greek} χθόνιος\selectlanguage{english} (‘the god below’), and as \selectlanguage{greek} οὐράνιος\selectlanguage{english} (‘the god above’), since denial of burial pollutes the realm of \selectlanguage{greek} οἱ ἄνω θεοί\selectlanguage{english}, ‘the gods above.’” (450). We omit the observances of common, decent human behaviour at our peril. These laws are natural, unwritten and eternal; they are Antigone’s defense, for burial is central in the Antigone, as it is in the Ajax. Aristotle includes these lines of Antigone in his discussion of the Unwritten Laws in his Rhetoric 1.13.2. This question of laws, unwritten and man-made has occupied us ever since, and we still have not really reached a solid conclusive answer \cite{Collins}.} οὐ γάρ τί μοι Ζεὺς ἦν ὁ κηρύξας τάδε, \\
οὐδ᾽ ἡ ξύνοικος τῶν κάτω θεῶν Δίκη \\
τοιούσδ᾽ ἐν ἀνθρώποισιν ὥρισεν νόμους. \\
οὐδὲ σθένειν τοσοῦτον ᾠόμην τὰ σὰ \\
κηρύγμαθ᾽, ὥστ᾽ ἄγραπτα κἀσφαλῆ θεῶν \\
νόμιμα δύνασθαι θνητὸν ὄνθ᾽ ὑπερδραμεῖν. \\
οὐ γάρ τι νῦν γε κἀχθές, ἀλλ᾽ ἀεί ποτε \\
ζῇ ταῦτα, κοὐδεὶς οἶδεν ἐξ ὅτου 'φάνη. \\
τούτων ἐγὼ οὐκ ἔμελλον, ἀνδρὸς οὐδενὸς \\
φρόνημα δείσασ᾽, ἐν θεοῖσι τὴν δίκην \\
δώσειν: θανουμένη γὰρ ἐξῄδη, τί δ᾽ οὔ; \\
κεἰ μὴ σὺ προὐκήρυξας. εἰ δὲ τοῦ χρόνου \\
πρόσθεν θανοῦμαι, κέρδος αὔτ᾽ ἐγὼ λέγω. \\
ὅστις γὰρ ἐν πολλοῖσιν ὡς ἐγὼ κακοῖς \\
ζῇ, πῶς ὅδ᾽ Οὐχὶ κατθανὼν κέρδος φέρει;  \\
οὕτως ἔμοιγε τοῦδε τοῦ μόρου τυχεῖν \\
παρ᾽ οὐδὲν ἄλγος: ἀλλ᾽ ἄν, εἰ τὸν ἐξ ἐμῆς \\
μητρὸς θανόντ᾽ ἄθαπτον ἠνσχόμην νέκυν, \\
κείνοις ἂν ἤλγουν: τοῖσδε δ᾽ οὐκ ἀλγύνομαι. \\
σοὶ δ᾽ εἰ δοκῶ νῦν μῶρα δρῶσα τυγχάνειν, \\
σχεδόν τι μώρῳ μωρίαν ὀφλισκάνω.  \\
\end{verse}

\maketitle
\selectlanguage{english}
\section{Traduzione}
\textsc{Creon} \\
You, you with your face bent to the ground, do you admit, or deny that you did this? \\
\textsc{Antigone} \\
I declare it and make no denial. \\
\textsc{Creon} \\
\textit{(To the Guard)} You can take yourself wherever you please, [445] free and clear of a heavy charge. \textit{(Exit Guard. To Antigone)} You, however, tell me — not at length, but briefly — did you know that an edict had forbidden this? \\
\textsc{Antigone} \\
I knew it. How could I not? It was public. \\
\textsc{Creon} \\
And even so you dared overstep that law? \\
\textsc{Antigone} \\
Yes [450], since it was not Zeus that published me that edict, and since not of that kind are the laws which Justice who dwells with the gods below established among men. Nor did I think that your decrees were of such force, that a mortal could override the unwritten [455] and unfailing statutes given us by the gods. For their life is not of today or yesterday, but for all time, and no man knows when they were first put forth. Not for fear of any man's pride was I about to owe a penalty to the gods for breaking these. [460] Die I must, that I knew well (how could I not?). That is true even without your edicts. But if I am to die before my time, I count that a gain. When anyone lives as I do, surrounded by evils, how can he not carry off gain by dying? [465] So for me to meet this doom is a grief of no account. But if I had endured that my mother's son should in death lie an unburied corpse, that would have grieved me. Yet for this, I am not grieved. And if my present actions are foolish in your sight, [470] it may be that it is a fool who accuses me of folly.


\medskip

\printbibliography[title={Bibliografia essenziale}]

\end{document}
